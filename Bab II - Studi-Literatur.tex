% ==========================================
% BAB II STUDI LITERATUR
% ==========================================
\chapter{STUDI LITERATUR}
\label{chap:studi-literatur}

\section{Kukang Sunda (\textit{Nycticebus coucang})}

Kukang Sunda (\textit{Nycticebus coucang}) merupakan salah satu dari sembilan spesies kukang yang termasuk dalam genus \textit{Nycticebus}. Primata ini berada dalam famili Lorisidae dan ordo Primates, dengan ciri khas berupa tubuh kecil, gerakan lambat, mata besar, serta adaptasi kuat terhadap kehidupan nokturnal. Secara taksonomi, spesies ini dikategorikan sebagai bagian dari subordo Strepsirrhini, kelompok primata dengan karakteristik morfologis seperti rhinarium basah dan indra penciuman yang dominan \autocite{nekaris2007unexpected}.

Secara geografis, Kukang Sunda tersebar di kawasan Asia Tenggara, khususnya di Sumatra, Semenanjung Malaysia, Thailand Selatan, dan sebagian kecil wilayah Singapura. Habitat utamanya meliputi hutan hujan tropis, hutan sekunder, serta kawasan terdegradasi yang masih menyediakan tutupan vegetasi yang cukup. Kajian lapangan menunjukkan bahwa Kukang Sunda masih dapat bertahan di wilayah yang mengalami gangguan moderat, meskipun preferensi utamanya adalah hutan dengan struktur kanopi yang rapat untuk mendukung mobilitas arborealnya \autocite{neprimate2024}.

Distribusi yang luas ini tidak menjamin kelestarian populasinya karena tekanan antropogenik, terutama deforestasi dan fragmentasi habitat. Studi-studi konservasi menegaskan bahwa populasi Kukang Sunda mengalami penurunan signifikan dalam tiga dekade terakhir, sehingga pemahaman mengenai distribusi aktual dan kondisi habitat menjadi krusial untuk upaya perlindungan yang efektif \autocite{neprimate2024}.

\subsection{Perilaku, Ekologi, dan Pola Aktivitas}

Kukang Sunda merupakan primata nokturnal dengan aktivitas utama terjadi setelah matahari terbenam. Perilakunya cenderung soliter, bergerak perlahan, dan menunjukkan pola aktivitas yang sangat bergantung pada ketersediaan sumber pakan. Hasil penelitian lapangan menunjukkan bahwa individu kukang dapat menghabiskan hingga 40\% waktunya untuk mencari makan, dengan pola jelajah yang relatif terbatas namun konsisten di wilayah teritorialnya \autocite{neprimate2024}.

Secara ekologis, Kukang Sunda memainkan peran penting dalam menjaga keberlanjutan ekosistem hutan tropis. Dietnya yang beragam seperti buah, nektar, getah pohon, dan berbagai jenis serangga menempatkannya sebagai agen penyebar biji (seed disperser), penyerbuk (pollinator), sekaligus pengendali populasi serangga di habitat alaminya. Perannya sebagai penyerbuk, misalnya, terkait dengan kemampuannya mengakses bunga pada malam hari, memberikan kontribusi signifikan terhadap regenerasi vegetasi dan dinamika struktur hutan \autocite{neprimate2024}.

Namun demikian, karakteristik perilaku kukang yang lambat, pemalu, dan aktif di ketinggian kanopi membuat pengamatannya menjadi tantangan besar. Aktivitas nokturnal menyebabkan rendahnya visibilitas, sedangkan posisi tubuh yang sering bersembunyi di balik dedaunan membuat pengenalan visual sangat bergantung pada teknologi dan teknik dokumentasi yang memadai \autocite{neprimate2024}.

\subsection{Status Konservasi dan Ancaman}

Berdasarkan daftar merah International Union for Conservation of Nature (IUCN), Kukang Sunda dikategorikan sebagai spesies berstatus Terancam Punah (Endangered) \autocite{iucn2020}. Status ini merefleksikan penurunan populasi yang diperkirakan mencapai 30–50\% dalam tiga generasi, terutama akibat degradasi habitat, fragmentasi kawasan hutan, serta tekanan antropogenik yang terus meningkat. Penelitian konservasi sebelumnya menunjukkan bahwa kombinasi deforestasi, hilangnya tutupan kanopi, dan semakin sempitnya ruang jelajah telah memberikan dampak signifikan terhadap keberlangsungan populasi kukang di alam liar \autocite{neprimate2024}.

Salah satu ancaman paling serius yang dihadapi Kukang Sunda adalah perdagangan ilegal. Laporan konservasi terbaru yang disusun oleh Yayasan IAR Indonesia mengungkapkan bahwa kukang secara rutin ditangkap untuk dijadikan hewan peliharaan atau objek hiburan, terutama karena meningkatnya permintaan pasar dan penyebaran konten digital yang menampilkan kukang sebagai hewan lucu \autocite{iar2017trade}. Banyak individu kukang mengalami perlakuan brutal selama proses penangkapan dan distribusi, termasuk praktik pencabutan gigi yang bertujuan menghilangkan kemampuan menggigitnya, padahal kukang memiliki kelenjar racun. Tindakan tersebut tidak hanya melanggar prinsip kesejahteraan hewan, tetapi juga menyebabkan tingkat mortalitas yang sangat tinggi, baik sebelum maupun sesudah masuk ke rantai perdagangan.

Selain perdagangan ilegal, fragmentasi hutan akibat konversi lahan untuk pertanian, perkebunan, dan pemukiman turut memperburuk kondisi konservasi. Hilangnya struktur kanopi yang kontinu sebagai jalur utama pergerakan kukang membatasi kemampuan mereka berpindah, mencari makan, dan berinteraksi dengan kelompok populasi lain. Isolasi populasi yang terjadi dalam jangka panjang dapat menurunkan keragaman genetik dan meningkatkan kerentanan terhadap ancaman eksternal, menjadikan upaya konservasi semakin mendesak.

Dengan besarnya tekanan ekologis dan antropogenik tersebut, pemantauan populasi kukang yang lebih akurat, efisien, dan didukung teknologi menjadi kebutuhan mendesak dalam rangka meningkatkan efektivitas program perlindungan dan rehabilitasi.

\section{Ekspresi Wajah pada Primata}
Ekspresi wajah merupakan salah satu kanal komunikasi sosial terpenting pada primata. Banyak spesies primata menggunakan perubahan otot wajah untuk menyampaikan emosi, intensi perilaku, serta status sosial dalam kelompok. Menurut \textcite{burrows2008facial}, primata umumnya memiliki sistem musculi mimetis (facial musculature) yang lebih kompleks dibanding banyak mamalia lain, yang memungkinkan rentang ekspresi wajah cukup luas tergantung adaptasi ekologis dan kondisi sosial.

\subsection{Sistem Pengkodean Wajah (FACS dan AnimalFACS)}

Untuk menganalisis ekspresi wajah secara objektif dan terstandarisasi, metode yang paling umum digunakan adalah \textit{Facial Action Coding System} (FACS). Dikembangkan pertama kali oleh Ekman dan Friesen pada tahun 1978 untuk manusia, sistem ini berbasis anatomi yang mengkategorikan gerakan wajah berdasarkan kontraksi otot individu atau kelompok otot yang disebut sebagai \textit{Action Units} (AUs) \autocite{ekman1978}. Keunggulan utama FACS adalah objektivitasnya; sistem ini mendeskripsikan gerakan fisik otot semata tanpa mengasumsikan emosi di baliknya, sehingga menghindari bias interpretasi pengamat \autocite{waller2013}.

Mengingat kedekatan evolusioner dan kemiripan struktur otot wajah, FACS telah diadaptasi ke berbagai spesies primata non-manusia, yang dikenal sebagai AnimalFACS. Beberapa adaptasi yang telah mapan antara lain:
\begin{enumerate}
    \item ChimpFACS untuk simpanse (\textit{Pan troglodytes}), yang memetakan gerakan wajah homolog dengan manusia \autocite{vick2007}.
    \item MaqFACS untuk monyet rhesus (\textit{Macaca mulatta}), yang juga mengidentifikasi gerakan unik spesies seperti \textit{Ear Flapping} (EAU1) yang tidak ditemukan pada manusia \autocite{parr2010}.
\end{enumerate}
Adaptasi ini memungkinkan peneliti untuk membandingkan repertoar ekspresi antar spesies (komparatif) dan menerjemahkan observasi biologis menjadi data komputasional yang terstruktur.

\subsection{Anatomi Wajah dan Potensi Ekspresi pada Kukang}

Penerapan analisis ekspresi wajah pada Kukang Sunda menghadapi tantangan taksonomi, mengingat posisinya sebagai primata \textit{Strepsirrhini} (berhidung basah). Kelompok ini secara tradisional dianggap memiliki mobilitas wajah yang lebih terbatas dibandingkan \textit{Haplorhini} (primata berhidung kering seperti monyet dan kera) karena struktur bibir atas yang menyatu dengan rhinarium \autocite{diogo2012}.

Namun, penelitian anatomi terbaru mengubah paradigma ini. Studi komprehensif yang dilakukan oleh \textcite{burrows2024masks} melalui diseksi dan pemindaian 3D pada genus \textit{Nycticebus} berhasil memetakan 19 otot wajah, termasuk otot-otot halus seperti \textit{depressor anguli oris} dan \textit{constrictor nasalis} \autocite{burrows2024masks}. Temuan ini mengonfirmasi bahwa kukang memiliki "perangkat keras" anatomis yang memadai untuk menghasilkan ekspresi wajah yang kompleks dan bernuansa.

Selain itu, pola pewarnaan wajah (\textit{facial mask}) pada kukang yang memiliki kontras tinggi—seperti cincin gelap di sekitar mata—diduga berfungsi sebagai sinyal aposematik untuk menarik perhatian pada kemampuan gigitan berbisa mereka \autocite{nekaris2019}. Fitur kontras tinggi ini secara teknis menguntungkan dalam pengembangan sistem deteksi berbasis \textit{Computer Vision} karena menciptakan gradien piksel yang tajam pada citra digital.

\subsection{Grimace Scale sebagai Indikator Kesejahteraan}

Dalam konteks konservasi dan pemantauan kesejahteraan hewan, analisis ekspresi wajah sering difokuskan pada deteksi nyeri atau stres melalui metode \textit{Grimace Scale}. Berbeda dengan FACS yang memetakan seluruh gerakan, \textit{Grimace Scale} secara spesifik mengidentifikasi perubahan fitur wajah yang berkorelasi dengan ketidaknyamanan fisik \autocite{keating2012}.

Meskipun skala resmi untuk kukang belum dibakukan, literatur etologi mengidentifikasi beberapa perubahan wajah spesifik yang menjadi indikator stres atau nyeri pada spesies ini:
\begin{enumerate}
    \item \textbf{\textit{Orbital Tightening}:} Penutupan kelopak mata atau mata yang menyipit (\textit{squinting}) bukan karena kantuk, melainkan respons terhadap nyeri atau cahaya terang \autocite{nekaris2022}.
    \item \textbf{Posisi Telinga:} Telinga yang ditarik ke belakang dan mendatar terhadap kepala (\textit{flattened ears}) merupakan indikator universal agresi defensif atau ketidaknyamanan pada banyak mamalia, termasuk kukang \autocite{fuller2015}.
    \item \textbf{Perubahan Mulut:} Menyeringai atau menarik sudut bibir sering kali merupakan tanda ketakutan, sementara menunjukkan gigi (\textit{baring teeth}) adalah sinyal ancaman yang jelas \autocite{nekaris2022}.
\end{enumerate}

Pemahaman terhadap indikator visual ini menjadi dasar penting dalam pelabelan dataset untuk pelatihan model kecerdasan buatan, memungkinkan sistem untuk membedakan antara kondisi "Netral" dan "Stres/Sakit" secara otomatis.

\section{Teknologi Computer Vision dalam Pemantauan Satwa Liar}

Pemanfaatan teknologi dalam konservasi telah bergeser dari metode manual menuju otomatisasi cerdas. Secara tradisional, pemantauan populasi dan perilaku satwa liar bergantung pada transek garis dan pengamatan langsung yang memiliki keterbatasan signifikan, seperti biaya tinggi, intensitas tenaga kerja yang besar, serta potensi gangguan terhadap perilaku alami hewan (\textit{observer effect}) \autocite{greenberg2020}.

Revolusi penggunaan kamera trap digital (camera trap) menghasilkan lonjakan data visual yang tidak mungkin lagi dianalisis secara manual. Oleh karena itu, \textit{Computer Vision} (CV)—cabang kecerdasan buatan yang melatih komputer untuk menafsirkan citra—telah menjadi standar baru untuk mengotomatisasi analisis ekologi, mulai dari deteksi spesies, penghitungan individu, hingga pengenalan perilaku \autocite{norouzzadeh2018}.

\subsection{Deteksi Objek: Arsitektur YOLO}

Langkah fundamental dalam sistem analisis wajah hewan adalah melokalisasi area wajah di dalam \textit{frame} gambar yang luas. Untuk tugas ini, algoritma deteksi objek berbasis \textit{Deep Learning} menjadi pilihan utama. Di antara berbagai arsitektur yang ada, keluarga YOLO (\textit{You Only Look Once}) mendominasi implementasi \textit{real-time} karena efisiensinya.

Berbeda dengan metode dua tahap (\textit{two-stage}) seperti Faster R-CNN yang memisahkan proses proposal wilayah dan klasifikasi, YOLO memprediksi kotak pembatas (\textit{bounding boxes}) dan probabilitas kelas secara simultan dalam satu kali pemrosesan Neural Network \autocite{yolobat2020}.

\begin{itemize}
    \item \textbf{YOLOv5:} Versi ini sering digunakan dalam studi satwa liar karena keseimbangan optimal antara kecepatan inferensi dan akurasi, serta ukurannya yang ringan untuk perangkat lapangan. \textcite{lei2022postural} membuktikan efektivitas YOLOv5 dalam mendeteksi dan mengklasifikasikan perilaku postural Kukang Bengal (\textit{Nycticebus bengalensis}) dengan presisi tinggi pada kondisi malam hari \autocite{lei2022postural}.
    \item \textbf{YOLOv8 Attention Mechanisms:} Iterasi terbaru seperti YOLOv8 menawarkan peningkatan pada mekanisme \textit{Feature Pyramid Network} (FPN) untuk mendeteksi objek kecil. Integrasi modul atensi (seperti CBAM) pada arsitektur ini terbukti meningkatkan kemampuan deteksi pada citra termal/inframerah dengan menekan \textit{noise} dari latar belakang vegetasi yang kompleks \autocite{nightyolo2024}.
\end{itemize}

\subsection{Deep Learning untuk Klasifikasi Ekspresi}

Setelah wajah terdeteksi dan di-\textit{crop}, tahap selanjutnya adalah klasifikasi ekspresi (misalnya: Netral vs. Stres). \textit{Convolutional Neural Networks} (CNN) merupakan arsitektur yang paling efektif untuk tugas ini karena kemampuannya mengekstraksi fitur visual secara hierarkis, mulai dari tepi sederhana hingga tekstur wajah yang kompleks \autocite{facialsurvey2023}.

Kendala utama dalam penerapan CNN pada satwa liar langka seperti Kukang Sunda adalah kelangkaan data (\textit{data scarcity}). Melatih model dari awal membutuhkan ribuan citra berlabel. Solusi standar untuk masalah ini adalah \textit{Transfer Learning}, yaitu teknik memanfaatkan model yang telah dilatih pada dataset raksasa (seperti ImageNet atau dataset wajah manusia) dan mengadaptasinya untuk tugas baru.

Studi terbaru, seperti inisiatif \textit{PrimateFace}, menunjukkan bahwa model yang telah "belajar" mengenali wajah manusia memiliki performa dasar yang baik pada wajah primata, yang kemudian dapat ditingkatkan secara drastis melalui proses \textit{fine-tuning} spesifik spesies \autocite{primateface2025}.

\subsection{Pengolahan Citra Nokturnal (\textit{Night Vision})}

Karena sifat nokturnal Kukang Sunda, sebagian besar data diperoleh menggunakan kamera inframerah (IR). Citra IR memiliki karakteristik unik yang menantang bagi model standar yang dilatih pada citra RGB siang hari:
\begin{enumerate}
    \item {Kontras Rendah:} Citra cenderung datar dengan histogram sempit, menyulitkan pembedaan fitur wajah \autocite{infrared2024}.
    \item {Distribusi Cahaya Tidak Merata:} Objek yang dekat kamera mengalami \textit{overexposure} ("efek senter"), sementara latar belakang menjadi hitam pekat \autocite{greenberg2020}.
\end{enumerate}

Untuk mengatasi hal ini, teknik pra-pemrosesan citra (\textit{preprocessing}) seperti \textit{Contrast Limited Adaptive Histogram Equalization} (CLAHE) sangat krusial. Berbeda dengan ekualisasi histogram biasa yang memperkuat \textit{noise}, CLAHE meningkatkan kontras lokal secara adaptif. Penelitian oleh \textcite{tiwari2023enhancing} menunjukkan bahwa penerapan CLAHE pada citra malam secara signifikan meningkatkan \textit{Mean Average Precision} (mAP) algoritma deteksi karena mampu menonjolkan tepi dan detail tekstur yang sebelumnya tersembunyi dalam kegelapan \autocite{tiwari2023enhancing}.

\section{Dataset Satwa}

Salah satu hambatan terbesar dalam penerapan \textit{Deep Learning} untuk konservasi satwa liar adalah ketersediaan dataset terlabel yang berkualitas. Berbeda dengan pengenalan wajah manusia yang didukung oleh dataset masif (seperti LFW atau CelebA), dataset untuk primata, terlebih lagi untuk spesies nokturnal yang terancam punah, sangat terbatas (\textit{data scarcity}).

\section{Literatur Terkait}

\subsection{Penerapan Pengenalan Wajah pada Primata: Studi Kasus LemurFaceID}

Salah satu penelitian fundamental dalam biometrik primata non-manusia dilakukan oleh \textcite{crouse2017lemur} yang mengembangkan sistem \textit{LemurFaceID} untuk identifikasi individu Lemur Perut Merah (\textit{Eulemur rubriventer}) di Ranomafana National Park, Madagaskar. Penelitian ini dilatarbelakangi oleh keterbatasan metode konvensional seperti penandaan fisik (\textit{tagging/collaring}) yang bersifat invasif, berisiko melukai hewan, dan mahal, serta metode identifikasi manual yang sangat bergantung pada keahlian pengamat dan rentan terhadap bias subjektif.

Dalam pengembangan \textit{LemurFaceID}, \textcite{crouse2017lemur} menemukan tantangan teknis yang spesifik pada wajah hewan yang tertutup bulu. Eksperimen awal mereka menggunakan fitur \textit{Scale Invariant Feature Transform} (SIFT) yang umum digunakan pada objek kaku dan menghasilkan akurasi yang rendah (Rank-1: 73\%). Hal ini disebabkan oleh sifat fitur SIFT yang terlalu sensitif terhadap pola rambut/bulu lokal yang berubah-ubah antar citra akibat faktor lingkungan atau pergerakan.

Sebagai solusi, penelitian ini beralih menggunakan pendekatan berbasis tekstur dengan metode \textit{Patch-wise Multiscale Local Binary Pattern} (MLBP). Alur pemrosesan citra yang diterapkan meliputi:
\begin{enumerate}
    \item {Pra-pemrosesan (\textit{Preprocessing}):} Penyelarasan mata (\textit{eye alignment}) secara manual, pemotongan (\textit{cropping}), dan konversi ke \textit{grayscale}.
    \item {Normalisasi Iluminasi:} Mengadaptasi metode Tan dan Triggs untuk mengurangi dampak variasi pencahayaan dan tekstur bulu halus, menggunakan filter \textit{Difference of Gaussians} (DoG).
    \item {Ekstraksi Fitur:} Penggunaan MLBP untuk menangkap pola tekstur wajah lokal.
    \item {Reduksi Dimensi:} Penerapan \textit{Linear Discriminant Analysis} (LDA) untuk memaksimalkan perbedaan antar-kelas (antar individu) dan meminimalkan variasi dalam-kelas (foto berbeda dari individu yang sama).
\end{enumerate}

Evaluasi sistem dilakukan menggunakan dataset berisi 462 citra dari 80 individu. Hasil pengujian menunjukkan tingkat akurasi yang sangat tinggi, mencapai 98,7\% ($\pm$ 1,81\%) pada skenario \textit{closed-set} dengan menggunakan fusi dua citra kueri.

Temuan ini sangat relevan bagi pengembangan sistem deteksi pada primata lain yang memiliki variasi pola wajah (masker wajah) yang distingtif, seperti Kukang Sunda. \textcite{crouse2017lemur} menyimpulkan bahwa teknologi pengenalan wajah mampu menjadi alternatif non-invasif yang valid untuk studi longitudinal jangka panjang, mengatasi kendala logistik dan etika yang melekat pada metode penandaan fisik.

\subsection{Era Deep Learning pada Video Satwa Liar: Studi Kasus Simpanse}

Jika \textcite{crouse2017lemur} meletakkan dasar pengenalan wajah primata menggunakan fitur tekstur manual, penelitian oleh \textcite{schofield2019chimpanzee} menandai lompatan teknologi menuju penggunaan \textit{Deep Learning} secara menyeluruh (\textit{end-to-end}). Dalam studi mereka mengenai pengenalan wajah simpanse (\textit{Pan troglodytes}) di alam liar, Schofield et al. mengembangkan sistem yang mampu menangani tantangan visual ekstrem yang sering gagal ditangani oleh metode konvensional, seperti pencahayaan buruk, gerakan cepat (\textit{motion blur}), dan oklusi sebagian


Kontribusi utama dari penelitian ini adalah penerapan arsitektur Artificial Neural Network (ANN) yang kompleks pada data video, bukan hanya citra diam (\textit{still images}). Pipa pemrosesan yang dikembangkan terdiri dari beberapa tahap:
\begin{enumerate}
    \item {Deteksi Wajah (\textit{Single Shot Detector}/SSD):} Menggunakan model SSD untuk mendeteksi wajah simpanse dalam setiap \textit{frame} video secara \textit{real-time}.
    \item {Pelacakan Wajah (\textit{Tracking}):} Menerapkan pelacak Kanade-Lucas-Tomasi untuk menghubungkan deteksi wajah antar-\textit{frame} menjadi satu kesatuan lintasan (\textit{tracklet}). Hal ini memungkinkan sistem untuk mengagregasi informasi dari berbagai sudut pandang wajah dalam satu urutan video.
    \item {Pengenalan Identitas (CNN):} Menggunakan arsitektur ResNet-50 (VGG-M) yang telah dilatih ulang (\textit{fine-tuned}) untuk mengklasifikasikan identitas individu berdasarkan fitur wajah yang diekstraksi dari CNN.
\end{enumerate}

Salah satu faktor kunci keberhasilan \textcite{schofield2019chimpanzee} adalah volume data. Mereka membangun arsip yang berisi lebih dari 10 juta deteksi wajah dari ribuan jam rekaman video yang dikumpulkan selama 14 tahun di Guinea, Afrika Barat. Evaluasi sistem menunjukkan akurasi identifikasi sebesar 92,5\%, sebuah pencapaian signifikan mengingat data yang digunakan adalah data mentah "di alam liar" tanpa kontrol pencahayaan atau pose.

Studi ini memberikan implikasi penting bagi penelitian Kukang Sunda, terutama dalam hal penggunaan data video dibandingkan foto statis. Pendekatan Schofield membuktikan bahwa dengan memanfaatkan informasi temporal (urutan frame dalam video), model AI dapat mengenali individu dengan lebih akurat meskipun kualitas per-frame rendah akibat kondisi lingkungan (misalnya, kondisi malam hari pada kukang).

\section{\textit{Research Gap}}

Berdasarkan tinjauan terhadap literatur terdahulu dan kondisi lapangan, penelitian ini mengidentifikasi beberapa kesenjangan krusial yang menghambat penerapan teknologi \textit{Deep Learning} pada konservasi Kukang Sunda. Kesenjangan ini dikategorikan ke dalam tiga aspek utama:

\begin{enumerate}
    \item {Kelangkaan dan Ketiadaan Standardisasi Dataset (\textit{Data Scarcity})} \\
    Berbeda dengan penelitian pengenalan wajah manusia atau primata diurnal (seperti simpanse dan makaka) yang didukung oleh dataset publik berskala besar (misalnya \textit{PrimateFace} atau \textit{Labeled Faces in the Wild}), penelitian pada primata nokturnal menghadapi kendala \textit{data scarcity} yang ekstrem.
    \begin{itemize}
        \item Belum tersedia dataset publik yang teranotasi secara baku untuk Kukang Sunda, sehingga validasi silang antar-penelitian sulit dilakukan.
        \item Data yang ada umumnya terbatas pada lingkungan terkontrol (pusat rehabilitasi) dan kurang merepresentasikan variabilitas kondisi alam liar (\textit{unconstrained environments}).
        \item Keterbatasan jumlah sampel ini meningkatkan risiko \textit{overfitting} pada model \textit{Deep Learning} konvensional jika tidak ditangani dengan strategi augmentasi yang tepat.
    \end{itemize}

    \item {Degradasi Kualitas Citra pada Domain Nokturnal} \\
    Mayoritas model deteksi objek (seperti YOLO) dilatih menggunakan citra RGB dengan pencahayaan optimal. Penerapan langsung model tersebut pada citra kukang menghadapi \textit{domain shift} yang signifikan karena:
    \begin{itemize}
        \item Citra diambil menggunakan kamera \textit{night vision}/inframerah yang bersifat monokromatik dan sering kali memiliki \textit{noise} tinggi (ISO tinggi).
        \item Efek \textit{motion blur} sering terjadi akibat pergerakan hewan dalam kondisi \textit{low-light} dengan \textit{shutter speed} rendah.
        \item Tanpa teknik pra-pemrosesan (\textit{preprocessing}) khusus, fitur visual penting sering kali hilang dalam kegelapan atau tertutup \textit{noise}, menurunkan akurasi deteksi secara drastis.
    \end{itemize}

    \item {Kompleksitas Pengenalan Fitur Ekspresi Halus (\textit{Subtle Features})} \\
    Tantangan biometrik pada kukang jauh lebih tinggi dibandingkan manusia karena variansi morfologi antar-individu yang sangat rendah (sulit dibedakan secara kasat mata). Selain itu, ekspresi wajah kukang sebagai indikator stres atau kesejahteraan tidak menunjukkan perubahan bentuk wajah yang drastis, melainkan melalui sinyal mikro (\textit{micro-signals}) seperti:
    \begin{itemize}
        \item Perubahan intensitas pembukaan mata (\textit{orbital tightening}).
        \item Perubahan posisi telinga dan postur tubuh yang perlahan.
    \end{itemize}
\end{enumerate}
    Model \textit{Facial Expression Recognition} (FER) standar cenderung gagal menangkap nuansa ini tanpa mekanisme atensi visual (\textit{visual attention}) atau resolusi citra yang memadai.
