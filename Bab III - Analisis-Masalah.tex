% ============================================================================================
% BAB III ANALISIS MASALAH
% Pembagian subbab tidak rigid dan dapat bervariasi. Bab ini minimal berisi analisis kebutuhan
% fungsional dan nonfungsional, analisis berbagai alternatif solusi yang dapat ditawarkan, dan
% metode pemilihan solusi yang diusulkan.
% ============================================================================================
\chapter{ANALISIS MASALAH}
\label{chap:analisis-masalah}

\section{Analisis Kondisi Saat Ini}

Berdasarkan pengamatan terhadap proses rehabilitasi Kukang Sunda saat ini, teridentifikasi beberapa permasalahan utama yang mempengaruhi efektivitas pemantauan kesejahteraan satwa:

\begin{enumerate}
    \item Keterbatasan Metode Observasi Manual \\
    Penilaian kondisi kesehatan dan perilaku kukang sangat bergantung pada keahlian subjektif petugas lapangan. Hal ini menyebabkan inkonsistensi data dan risiko kesalahan diagnosis, serta tidak memungkinkan dilakukan pemantauan 24 jam secara intensif tanpa mengganggu satwa.

    \item Kendala Biologis dan Lingkungan \\
    Sifat nokturnal kukang serta ekspresi wajahnya yang tidak mencolok (\textit{subtle}) dan sulit untuk diketahui menyulitkan pengamatan visual langsung oleh manusia, terutama pada malam hari saat satwa berada pada puncak aktivitasnya.

    \item Kesenjangan Teknologi \\
    Kamera pengawas yang tersedia di fasilitas rehabilitasi memiliki resolusi rendah dan/atau kualitas citra \textit{low-light} yang buruk. Keterbatasan resolusi ini menyebabkan fitur-fitur wajah kukang yang penting seperti posisi telinga, mata, dan mulut sulit dibedakan dan diekstraksi oleh model komputasi. Selain itu, belum tersedianya dataset referensi wajah kukang yang terstandarisasi untuk kondisi \textit{low-light} menjadi penghalang utama dalam pengembangan sistem deteksi otomatis.

    \item Urgensi Peningkatan Efisiensi \\
    Dengan meningkatnya kasus perdagangan ilegal dan keterbatasan sumber daya manusia (SDM), metode tradisional dinilai tidak lagi memadai untuk menangani volume kukang yang membutuhkan perawatan medis intensif. Dibutuhkan inovasi teknologi non-invasif untuk mempercepat deteksi stres dan tanda-tanda gangguan kesehatan lainnya.
\end{enumerate}

\section{Analisis Kebutuhan}
Berdasarkan kondisi saat ini dan permasalahan yang teridentifikasi, perlu dilakukan analisis kebutuhan yang mencakup identifikasi masalah pengguna, kebutuhan fungsional, serta kebutuhan non-fungsional dari sistem yang akan dikembangkan. Analisis ini dilakukan secara kolaboratif antara tim ITB (teknologi), IPB (perilaku dan ground truth), dan UNPAD (hormonal dan fisiologis).

\subsection{Identifikasi Masalah Pengguna}
Permasalahan dalam pemantauan Kukang Sunda tidak hanya bersifat teknis, tetapi juga berdampak langsung pada pemangku kepentingan yang terlibat dalam proses rehabilitasi dan konservasi. Identifikasi masalah spesifik berdasarkan peran pengguna adalah sebagai berikut:

\begin{enumerate}
    \item \textbf{Petugas Konservasi (\textit{Conservation Officers})} \\
    Petugas lapangan bertanggung jawab atas pemantauan harian, pemberian pakan, dan pembersihan kandang. Masalah utama yang dihadapi meliputi:
    \begin{itemize}
        \item \textbf{Beban Operasional Tinggi:} Peningkatan jumlah kukang sitaan tidak sebanding dengan jumlah petugas, sehingga pemantauan intensif pada setiap individu menjadi sulit dilakukan. Metode manual yang ada saat ini sangat memakan waktu (\textit{time-consuming}) dan tidak praktis untuk skala populasi yang besar.
        \item \textbf{Subjektivitas Pengamatan:} Penilaian perilaku (seperti tingkat aktivitas atau nafsu makan) rentan terhadap bias antar-pengamat (\textit{inter-observer error}), di mana penilaian satu petugas bisa berbeda dengan petugas lainnya.
        \item \textbf{Keterbatasan Pengawasan Malam Hari:} Sifat nokturnal kukang mengharuskan pengawasan dilakukan pada malam hari. Keterbatasan visibilitas dan kelelahan manusia menurunkan kewaspadaan petugas terhadap tanda-tanda anomali perilaku.
    \end{itemize}
    
    \item \textbf{Dokter Hewan (\textit{Veterinarians})} \\
    Tim medis berfokus pada kesehatan fisik, fisiologis, dan kesiapan rilis satwa. Kendala spesifik yang dihadapi adalah:
    \begin{itemize}
        \item \textbf{Keterbatasan Diagnosa Jarak Jauh:} Kamera CCTV yang ada saat ini umumnya memiliki resolusi rendah dan kualitas \textit{night vision} yang buruk. Akibatnya, fitur-fitur wajah penting untuk diagnosa klinis—seperti \textit{orbital tightening} (penyipitan mata), posisi telinga, atau kondisi mulut—menjadi kabur dan tidak dapat dianalisis dari rekaman video.
        \item \textbf{Dilema Metode Invasif:} Pemeriksaan fisik secara langsung (\textit{handling}) sering kali memicu respons stres fisiologis akut hingga risiko cedera pada hewan. Dokter hewan membutuhkan metode pemantauan non-invasif yang mampu memberikan data objektif mengenai tingkat stres tanpa perlu menangkap hewan tersebut.
        \item \textbf{Deteksi Dini yang Terhambat:} Tanda-tanda klinis pada kukang sering kali bersifat subtil. Tanpa bantuan teknologi yang mampu mendeteksi perubahan mikro pada wajah atau postur, intervensi medis sering kali terlambat dilakukan.
    \end{itemize}    
    
    \item \textbf{Peneliti Primata (\textit{Primate Researchers})} \\
    Peneliti membutuhkan data yang akurat dan terstandarisasi untuk studi perilaku dan konservasi jangka panjang. Masalah yang dihadapi meliputi:
    \begin{itemize}
        \item \textbf{\textit{Observer Effect}:} Kehadiran peneliti manusia secara fisik di lokasi observasi dapat mengubah perilaku alami hewan yang sedang diamati.
        \item \textbf{Kelangkaan Data Terstandarisasi (\textit{Data Scarcity}):} Belum adanya dataset publik untuk ekspresi wajah primata nokturnal menghambat pengembangan studi komparatif. Sebagian besar data jangka panjang sulit diintegrasikan karena perbedaan metode identifikasi manual yang digunakan antar-penelitian.
        \item \textbf{Inefisiensi Pengumpulan Data:} Pengumpulan data etogram secara manual sangat lambat dan rentan kesalahan, membatasi volume data yang dapat dianalisis untuk memahami pola perilaku spesies secara komprehensif.
    \end{itemize}
\end{enumerate}
\subsection{Kebutuhan Fungsional}
Sistem yang akan dikembangkan harus memiliki fungsionalitas sebagai berikut:
% File: table/func-req.tex
\begin{table}[H]
\centering
\caption{Spesifikasi Kebutuhan Fungsional Sistem}
\label{tab:kebutuhan_fungsional}
\renewcommand{\arraystretch}{0.95}
% Definisi 3 kolom: l (rata kiri), l (rata kiri), X (rata kiri, auto-wrap)
\begin{tabularx}{\textwidth}{|l|p{2.5cm}|X|}
\hline
\textbf{Kode} & \textbf{Kebutuhan Fungsional} & \textbf{Deskripsi Kebutuhan} \\ 
\hline
F01 & Deteksi Wajah Kukang & Sistem mampu melokalisasi area wajah Kukang Sunda (\textit{Nycticebus coucang}) secara akurat dalam berbagai variasi intensitas cahaya, termasuk pada kondisi minim cahaya (\textit{low-light}) menggunakan citra inframerah. \\ 
\hline
F02 & Klasifikasi Ekspresi Wajah & Sistem dapat mengklasifikasikan status ekspresi wajah ke dalam kategori indikator kesejahteraan yang telah ditentukan, meliputi: santai (netral), stres, takut, agresif, dan tidak responsif. \\ 
\hline
F03 & Ekstraksi Fitur Wajah & Sistem mampu melakukan pelacakan (\textit{tracking}) otomatis pada fitur wajah kunci (seperti mata, telinga, dan mulut) untuk menganalisis perubahan morfologis yang berkaitan dengan ekspresi. \\ 
\hline
F04 & Pemantauan Berkelanjutan & Sistem menyediakan kapabilitas pemantauan ekspresi secara kontinu (24/7) untuk mendukung observasi kondisi satwa secara \textit{real-time} maupun terjadwal. \\ 
\hline
F05 & Manajemen Arsip Data & Sistem secara otomatis menyimpan log hasil deteksi, rekaman video, penanda waktu (\textit{timestamp}), dan metadata terkait ke dalam basis data untuk keperluan audit dan evaluasi lanjutan. \\ 
\hline
F06 & Visualisasi Dasbor & Sistem memvisualisasikan hasil analisis data melalui antarmuka dasbor yang informatif, mencakup grafik tren perilaku, status terkini, dan representasi visual hasil deteksi ekspresi. \\ 
\hline
F07 & Manajemen Input/Output & Sistem mendukung fungsionalitas impor data video eksternal untuk analisis \textit{batch} serta ekspor hasil laporan analisis statistik dalam format standar yang dapat diunduh. \\ 
\hline
F08 & Kontrol Akses Pengguna & Sistem menerapkan mekanisme manajemen hak akses berbasis peran (\textit{Role-Based Access Control}) untuk membedakan otoritas antara petugas lapangan, administrator, dan peneliti. \\ 
\hline
\bottomrule
\end{tabularx}
\end{table}

\subsection{Kebutuhan Non-fungsional}
Selain kebutuhan fungsional, sistem juga harus memenuhi beberapa kebutuhan non-fungsional:
\begin{table}[H]
\centering
\caption{Spesifikasi Kebutuhan Non-Fungsional Sistem}
\label{tab:kebutuhan_nonfungsional}
\renewcommand{\arraystretch}{0.85}
% Definisi 3 kolom: l (rata kiri), p{2.5cm} (rata kiri, lebar tetap), X (rata kiri, auto-wrap)
\begin{tabularx}{\textwidth}{|l|p{2.5cm}|X|}
\hline
\textbf{Kode} & \textbf{Kebutuhan Non-Fungsional} & \textbf{Deskripsi Kebutuhan} \\ 
\hline
NF01 & Kinerja (\textit{Performance}) & Sistem harus mampu memproses aliran data video dengan laju 15–30 \textit{frames per second} (FPS) dan menyelesaikan deteksi ekspresi dalam waktu kurang dari 1 detik per \textit{frame}. \\ 
\hline
NF02 & Keandalan (\textit{Reliability}) & Sistem harus mampu beroperasi secara stabil 24/7 dengan tingkat akurasi deteksi dan klasifikasi ekspresi minimal antara 85\% hingga 90\%. \\ 
\hline
NF03 & Skalabilitas (\textit{Scalability}) & Sistem harus dirancang untuk mengakomodasi penambahan jumlah \textit{feed} kamera dan ekspansi ukuran dataset pelatihan tanpa mengalami penurunan kinerja pemrosesan yang signifikan. \\ 
\hline
NF04 & Keamanan (\textit{Security}) & Sistem wajib menerapkan mekanisme keamanan data yang kuat, mencakup enkripsi penyimpanan data, otentikasi pengguna, dan proteksi terhadap akses yang tidak sah. \\ 
\hline
NF05 & Kemudahan Penggunaan (\textit{Usability}) & Antarmuka pengguna harus dirancang secara sederhana dan intuitif sehingga mudah dioperasikan oleh petugas konservasi yang tidak memiliki latar belakang teknis yang mendalam. \\ 
\hline
NF06 & Kompatibilitas (\textit{Compatibility}) & Sistem harus kompatibel dengan berbagai jenis kamera inframerah (\textit{IR/night vision}) yang tersedia dan fleksibel untuk diterapkan baik pada lingkungan \textit{server} lokal maupun komputasi awan (\textit{cloud}). \\ 
\hline
NF07 & Keterawatan (\textit{Maintainability}) & Sistem harus dibangun secara modular untuk mempermudah pemeliharaan, memungkinkan pembaruan model \textit{Artificial Intelligence} (AI) secara independen, dan memfasilitasi pelatihan ulang (\textit{retraining}) model. \\ 
\hline
NF08 & Portabilitas (\textit{Portability}) & Sistem harus dapat dijalankan pada lingkungan perangkat keras yang beragam, seperti komputer \textit{desktop}, unit \textit{server}, atau perangkat untuk pemantauan \textit{mobile}. \\ 
\hline
\bottomrule
\end{tabularx}
\end{table}

\section{Analisis Penentuan Solusi}
\label{sec:analisis_penentuan_solusi}

Analisis penentuan solusi dilakukan untuk memilih pendekatan terbaik dalam meningkatkan efektivitas pemantauan kondisi Kukang Sunda, terutama dalam mendeteksi ekspresi wajah secara otomatis dan non-invasif. Proses pemilihan dilakukan dengan membandingkan empat alternatif solusi menggunakan metode pembobotan berdasarkan kriteria evaluasi yang relevan dengan kebutuhan lapangan.

\subsection{Penentuan Kriteria Evaluasi}
Kriteria yang digunakan untuk menilai setiap alternatif solusi dijabarkan pada Tabel \ref{tab:kriteria_evaluasi} berikut:

\begin{table}[H]
\centering
\caption{Kriteria Evaluasi dan Pembobotan}
\label{tab:kriteria_evaluasi}
\renewcommand{\arraystretch}{1.3}
\begin{tabularx}{\textwidth}{@{}l c X@{}}
\toprule
\textbf{Kriteria} & \textbf{Bobot} & \textbf{Penjelasan} \\ 
\midrule
Akurasi \& Objektivitas & 0.30 & Kemampuan sistem mengurangi subjektivitas dan menghasilkan analisis ekspresi yang konsisten dan akurat. \\ 
Efisiensi Waktu & 0.20 & Kecepatan solusi dalam menghasilkan informasi dan mendukung pemantauan \textit{real-time}. \\ 
Keamanan Satwa (Non-Invasif) & 0.20 & Tingkat keamanan dan kenyamanan kukang, menghindari stres tambahan akibat interaksi fisik. \\ 
Biaya Implementasi & 0.15 & Tingkat biaya awal dan biaya pemeliharaan jangka panjang. \\ 
Kesesuaian dengan Kondisi Lapangan & 0.15 & Kesesuaian solusi dengan perilaku nokturnal kukang dan kondisi infrastruktur pusat rehabilitasi. \\ 
\bottomrule
\end{tabularx}
\end{table}

\subsection{Penilaian Alternatif Solusi}
Berdasarkan kriteria di atas, dilakukan penilaian terhadap empat alternatif solusi (AS). Hasil pembobotan disajikan dalam Tabel dibawah.

\begin{table}[H]
\centering
\caption{Matriks Penilaian Alternatif Solusi}
\label{tab:penilaian_alternatif}
\resizebox{\textwidth}{!}{%
\begin{tabular}{@{}l c c c c c c@{}}
\toprule
\textbf{Alternatif Solusi} & \textbf{Akurasi} & \textbf{Efisiensi} & \textbf{Keamanan} & \textbf{Biaya} & \textbf{Kesesuaian} & \textbf{Skor Total} \\
& (0.30) & (0.20) & (0.20) & (0.15) & (0.15) & \\ 
\midrule
AS01 – Observasi Manual & 2 & 2 & 4 & 4 & 2 & 2.55 \\ 
AS02 – Sensor Fisiologis (\textit{Wearable}) & 4 & 3 & 2 & 2 & 2 & 2.65 \\ 
AS03 – Pemantauan Video Tanpa AI & 3 & 3 & 5 & 4 & 3 & 3.55 \\ 
\textbf{AS04 – Sistem Deteksi Ekspresi Wajah Berbasis AI} & \textbf{5} & \textbf{5} & \textbf{5} & \textbf{3} & \textbf{5} & \textbf{4.70} \\ 
\bottomrule
\end{tabular}%
}
\end{table}

\noindent \textbf{Contoh Perhitungan untuk AS04:}

\subsection{Hasil Analisis}
Berdasarkan hasil pembobotan pada matriks keputusan di atas, diperoleh analisis sebagai berikut:
\begin{enumerate}
    \item \textbf{AS04 (Sistem Deteksi Ekspresi Wajah Berbasis AI \& Computer Vision)} memperoleh skor tertinggi dengan total \textbf{4.70}, mengungguli alternatif solusi lainnya secara signifikan.
    \item Solusi ini tidak hanya memenuhi kebutuhan prioritas utama (akurasi dan objektivitas), tetapi juga menjamin keamanan satwa (non-invasif) serta mampu bekerja secara \textit{real-time} pada lingkungan rendah cahaya (\textit{low-light}).
\end{enumerate}

\subsection{Keputusan Solusi yang Dipilih}
Berdasarkan analisis pembobotan yang telah dilakukan, solusi yang dipilih untuk dikembangkan dalam penelitian ini adalah \textbf{AS04}, yaitu:

\begin{quote}
\textbf{Sistem Deteksi dan Pengenalan Ekspresi Wajah Kukang Berbasis AI \& Computer Vision}
\end{quote}

Solusi ini dipilih karena alasan-alasan berikut:
\begin{enumerate}
    \item \textbf{Non-invasif:} Tidak menyebabkan stres pada satwa karena tidak memerlukan kontak fisik, sangat sesuai dengan karakter kukang yang sensitif.
    \item \textbf{Otomatisasi 24/7:} Dapat berjalan secara kontinu mengatasi keterbatasan jumlah SDM petugas di lapangan.
    \item \textbf{Akurasi Tinggi:} Memanfaatkan model \textit{Deep Learning} modern untuk meminimalkan subjektivitas penilaian manusia.
    \item \textbf{Efisiensi Biaya Jangka Panjang:} Meskipun investasi awal lebih tinggi dibanding observasi manual, solusi ini lebih hemat biaya operasional jangka panjang.
    \item \textbf{Dukungan Lingkungan Nokturnal:} Terintegrasi dengan kamera inframerah (\textit{Night Vision}) yang sesuai dengan habitat alami kukang.
    \item \textbf{Kualitas Data:} Menghasilkan data yang objektif, terstandardisasi, dan dapat didokumentasikan untuk keperluan penelitian lanjutan.
\end{enumerate}

Dengan demikian, AS04 ditetapkan sebagai pilihan terbaik untuk meningkatkan efektivitas konservasi dan pemantauan Kukang Sunda secara berkelanjutan.


