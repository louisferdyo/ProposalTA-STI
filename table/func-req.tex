% File: table/func-req.tex
\begin{table}[H]
\centering
\caption{Spesifikasi Kebutuhan Fungsional Sistem}
\label{tab:kebutuhan_fungsional}
\renewcommand{\arraystretch}{0.95}
% Definisi 3 kolom: l (rata kiri), l (rata kiri), X (rata kiri, auto-wrap)
\begin{tabularx}{\textwidth}{|l|p{2.5cm}|X|}
\hline
\textbf{Kode} & \textbf{Kebutuhan Fungsional} & \textbf{Deskripsi Kebutuhan} \\ 
\hline
F01 & Deteksi Wajah Kukang & Sistem mampu melokalisasi area wajah Kukang Sunda (\textit{Nycticebus coucang}) secara akurat dalam berbagai variasi intensitas cahaya, termasuk pada kondisi minim cahaya (\textit{low-light}) menggunakan citra inframerah. \\ 
\hline
F02 & Klasifikasi Ekspresi Wajah & Sistem dapat mengklasifikasikan status ekspresi wajah ke dalam kategori indikator kesejahteraan yang telah ditentukan, meliputi: santai (netral), stres, takut, agresif, dan tidak responsif. \\ 
\hline
F03 & Ekstraksi Fitur Wajah & Sistem mampu melakukan pelacakan (\textit{tracking}) otomatis pada fitur wajah kunci (seperti mata, telinga, dan mulut) untuk menganalisis perubahan morfologis yang berkaitan dengan ekspresi. \\ 
\hline
F04 & Pemantauan Berkelanjutan & Sistem menyediakan kapabilitas pemantauan ekspresi secara kontinu (24/7) untuk mendukung observasi kondisi satwa secara \textit{real-time} maupun terjadwal. \\ 
\hline
F05 & Manajemen Arsip Data & Sistem secara otomatis menyimpan log hasil deteksi, rekaman video, penanda waktu (\textit{timestamp}), dan metadata terkait ke dalam basis data untuk keperluan audit dan evaluasi lanjutan. \\ 
\hline
F06 & Visualisasi Dasbor & Sistem memvisualisasikan hasil analisis data melalui antarmuka dasbor yang informatif, mencakup grafik tren perilaku, status terkini, dan representasi visual hasil deteksi ekspresi. \\ 
\hline
F07 & Manajemen Input/Output & Sistem mendukung fungsionalitas impor data video eksternal untuk analisis \textit{batch} serta ekspor hasil laporan analisis statistik dalam format standar yang dapat diunduh. \\ 
\hline
F08 & Kontrol Akses Pengguna & Sistem menerapkan mekanisme manajemen hak akses berbasis peran (\textit{Role-Based Access Control}) untuk membedakan otoritas antara petugas lapangan, administrator, dan peneliti. \\ 
\hline
\bottomrule
\end{tabularx}
\end{table}